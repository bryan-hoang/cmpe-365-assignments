\documentclass[
  coursecode={CISC/CMPE 365},
  assignmentname={Lab 1},
  studentnumber=20053722,
  name={Bryan Hoang}
]{
  ltxanswer%
}

\marksnotpoints{}
\renewcommand{\questionlabel}{\textbf{Question:}}
\renewcommand{\questionshook}{
  \setlength{\leftmargin}{0pt}
  \setlength{\labelwidth}{-\labelsep}%
}
\renewcommand{\solutiontitle}{
  \noindent\textbf{Answer:}
}

\usepackage{bch-style}

\begin{document}
  \section*{Yes or no}
  \begin{questions}
    \question{}
    Is \(\log^{k}{n} \in \mathcal{O}(n), \forall k \ge 1\)?
    \begin{solution}
      Yes.
    \end{solution}
    \question{}
    Is \(\sqrt{n} \in \mathcal{O}(\log{n})\)?
    \begin{solution}
      No.
    \end{solution}
    \question{}
    Is \(n^{k} \in \mathcal{O}(2^{n}), \forall k \ge 1\)?
    \begin{solution}
      Yes.
    \end{solution}
  \end{questions}

  \section*{Problems and algorithms}
  Suppose all algorithms that solve a particular problem are in \(\Omega(n^{3})\).

  What is the strongest statement (of those below) that can be made about the fastest algorithm that solves this problem in time \(f(n)\)?
  \begin{questions}
    \question{}
    \(f(n) \in \mathcal{O}(n^{3})\)
    \begin{solution}
      Not necessarily true, since we don't know if the best algorithm will take \(n^{3}\) time. The upper bound also needs to be proved separately.
    \end{solution}
    \question{}
    \(f(n) \in \Theta(n^{3})\)
    \begin{solution}
      Can't claim this to be true, since we can't claim the previous statement for the reasons stated.
    \end{solution}
    \question{}
    \(f(n) \in \Omega(n^{3})\)
    \begin{solution}
      True, based on the premise of the question (i.e., \(f(n)\) is asymptotically lower bounded in \(n^{3}\) time)
    \end{solution}
  \end{questions}

  \section*{Tail recursion}
  \begin{questions}
    \question{}
    What is the tightest asymptotic complexity of
    \begin{equation*}
      T(n) = \begin{cases}
        c & \text{if}\ n \in \integers_{< 1}, \\
        0 & \text{otherwise}.
      \end{cases}
    \end{equation*}
    \begin{solution}
      \(T(n) \in \mathcal{O}(nf(n))\). Unless we know what \(f(n)\) is, we can't say for sure if \(T(n) \in \BigO(f(n))\). If \(f(n)\) is some constant, then we could say that \(T(n) \in \BigO(n)\).
    \end{solution}
  \end{questions}

  \section*{More general recursion}
  What is the tightest asymptotic complexity of
  \begin{equation*}
    T(n) = \begin{cases}
      c                   & \text{if}\ n \in \integers_{\le 1}, \\
      n + aT(\frac{n}{b}) & \text{otherwise}.
    \end{cases}
  \end{equation*}
  \begin{questions}
    \question{}
    When \((a, b) = (2, 4)\)?
    \begin{solution}
      \(T(n) \in \BigO()\).
    \end{solution}
    \question{}
    When \((a, b) = (3, 3)\)?
    \begin{solution}
      \(T(n) \in \BigO()\).
    \end{solution}
    \question{}
    When \((a, b) = (4, 2)\)?
    \begin{solution}
      \(T(n) \in \BigO()\).
    \end{solution}
  \end{questions}

  \section*{Tightest bound}
  \begin{questions}
    \question{}
    What is the tightest asymptotic complexity (from the list below) of
    \begin{equation*}
      T(n) = \begin{cases}
        c                   & \text{if}\ n \in \integers_{\le 2}, \\
        T(n - 1) + T(n - 2) & \text{otherwise}.
      \end{cases}
    \end{equation*}
    \begin{solution}
      \(T(n) \in \BigO()\).
    \end{solution}
  \end{questions}
\end{document}
